\documentclass[12pt]{article}
\usepackage{amsmath}
\usepackage{amssymb}
\usepackage{geometry}
\usepackage{enumitem}
\geometry{margin=1in}

\title{Triangular Numbers and Perfect Squares:\\A Problem for 2025}
\author{}
\date{}

\begin{document}

\maketitle

\section*{Problem Statement}

A \textbf{triangular number} has the form $T_n = \frac{n(n+1)}{2}$ for $n \in \mathbb{N}$.

\vspace{0.3cm}

\textbf{Problem:} What is the smallest perfect square of the form $T_{T_n + T_m}$ for $n, m \in \mathbb{N}$?

\vspace{1cm}

\section*{Solution}

\subsection*{Step 1: Understanding the Problem}

We need to find the smallest value of $T_{T_n + T_m}$ that is a perfect square. This means we're looking for the smallest $k$ such that:
\begin{equation}
T_{T_n + T_m} = k^2
\end{equation}
for some positive integers $n$ and $m$.

\subsection*{Step 2: Exploring Small Values}

Let's compute some small values of $T_{T_n + T_m}$:

\begin{align*}
T_{T_1 + T_1} &= T_{1 + 1} = T_2 = 3 \\
T_{T_1 + T_2} &= T_{1 + 3} = T_4 = 10 \\
T_{T_2 + T_2} &= T_{3 + 3} = T_6 = 21 \\
T_{T_1 + T_3} &= T_{1 + 6} = T_7 = 28 \\
T_{T_2 + T_3} &= T_{3 + 6} = T_9 = 45 \\
T_{T_3 + T_3} &= T_{6 + 6} = T_{12} = 78 \\
T_{T_1 + T_4} &= T_{1 + 10} = T_{11} = 66 \\
T_{T_2 + T_4} &= T_{3 + 10} = T_{13} = 91 \\
T_{T_3 + T_4} &= T_{6 + 10} = T_{16} = 136 \\
T_{T_4 + T_4} &= T_{10 + 10} = T_{20} = 210
\end{align*}

None of these values are perfect squares. We need to continue searching.

\subsection*{Step 3: Key Insight}

For $T_k$ to be a perfect square, we need $\frac{k(k+1)}{2} = s^2$ for some integer $s$. This is a well-known problem in number theory. The first few triangular numbers that are also perfect squares are:
\begin{align*}
T_1 &= 1 = 1^2 \\
T_8 &= 36 = 6^2 \\
T_{49} &= 1225 = 35^2 \\
T_{288} &= 41616 = 204^2
\end{align*}

However, we're not looking for any triangular perfect square; we need one of the form $T_{T_n + T_m}$.

\subsection*{Step 4: Finding the Answer}

We need $T_n + T_m$ to equal one of the indices where triangular numbers are perfect squares. Let's check if $T_n + T_m = 8$ is possible:
\begin{align*}
T_1 + T_3 &= 1 + 6 = 7 \quad (\text{close, but not } 8) \\
T_2 + T_2 &= 3 + 3 = 6 \quad (\text{not } 8)
\end{align*}

Since $8 = \frac{n(n+1)}{2}$ gives $n^2 + n - 16 = 0$, which has no integer solution, we cannot express 8 as a single triangular number. Can we express it as a sum of two triangular numbers?

Checking systematically:
\begin{align*}
T_1 + T_3 &= 1 + 6 = 7 \\
T_2 + T_2 &= 3 + 3 = 6 \\
T_1 + T_4 &= 1 + 10 = 11
\end{align*}

We cannot get 8 as $T_n + T_m$.

Let's try $T_n + T_m = 49$:
\begin{align*}
T_9 + T_1 &= 45 + 1 = 46 \\
T_8 + T_3 &= 36 + 6 = 42 \\
T_7 + T_5 &= 28 + 15 = 43 \\
T_6 + T_7 &= 21 + 28 = 49 \quad \checkmark
\end{align*}

So $T_{T_6 + T_7} = T_{49} = 1225 = 35^2$.

But wait! Let's check if there's something smaller we might have missed.

\subsection*{Step 5: A More Systematic Approach}

Actually, there's a beautiful pattern here. Notice that:
\begin{equation}
T_9 = 45 = \frac{9 \cdot 10}{2}
\end{equation}

And consider:
\begin{equation}
T_9 + T_9 = 45 + 45 = 90
\end{equation}

Now, $T_{90} = \frac{90 \cdot 91}{2} = \frac{8190}{2} = 4095$... wait, let me recalculate:
\begin{equation}
T_{90} = \frac{90 \cdot 91}{2} = 45 \cdot 91 = 4095
\end{equation}

Hmm, that's not a perfect square. Let me reconsider...

Actually, I need to be more careful. Let's note that $90 = 2 \cdot 45 = 2 \cdot T_9$.

But more importantly, observe that:
\begin{equation}
45^2 = 2025
\end{equation}

Is 2025 a triangular number? Let's check: we need $\frac{k(k+1)}{2} = 2025$, so $k(k+1) = 4050$.

Solving $k^2 + k - 4050 = 0$:
\begin{equation}
k = \frac{-1 + \sqrt{1 + 16200}}{2} = \frac{-1 + \sqrt{16201}}{2} = \frac{-1 + 127.33...}{2}
\end{equation}

This doesn't give an integer, so 2025 is not a triangular number in the usual sense.

\subsection*{Step 6: Correct Calculation}

Let me reconsider the problem more carefully. We have:
\begin{equation}
T_{90} = \frac{90 \cdot 91}{2} = 4095
\end{equation}

But actually, I should check: is there a value where $T_k = 45^2 = 2025$?

We need $\frac{k(k+1)}{2} = 2025$, so $k(k+1) = 4050$.

Let me try: $63 \cdot 64 = 4032$ (close), $64 \cdot 65 = 4160$ (too big).

So 2025 is not a triangular number.

\subsection*{Step 7: The Correct Answer}

Going back to our systematic search, we found that:
\begin{equation}
T_{T_6 + T_7} = T_{21 + 28} = T_{49} = \frac{49 \cdot 50}{2} = 1225 = 35^2
\end{equation}

Let's verify there's nothing smaller by checking all possibilities where $T_n + T_m < 49$:

For $T_k$ to be a perfect square with $k < 49$, we need $k \in \{1, 8\}$.

We already checked that $T_n + T_m = 8$ is impossible.

Therefore, the smallest perfect square of the form $T_{T_n + T_m}$ is:

\begin{equation}
\boxed{T_{T_6 + T_7} = T_{49} = 1225 = 35^2}
\end{equation}

\subsection*{Verification}

Let's verify our answer:
\begin{align*}
T_6 &= \frac{6 \cdot 7}{2} = 21 \\
T_7 &= \frac{7 \cdot 8}{2} = 28 \\
T_6 + T_7 &= 21 + 28 = 49 \\
T_{49} &= \frac{49 \cdot 50}{2} = \frac{2450}{2} = 1225 \\
\sqrt{1225} &= 35 \quad \checkmark
\end{align*}

Therefore, the answer is $\boxed{1225}$.

\end{document}
