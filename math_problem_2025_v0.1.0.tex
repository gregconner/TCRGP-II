\documentclass[12pt]{article}
\usepackage{amsmath}
\usepackage{amssymb}
\usepackage{geometry}
\geometry{margin=1in}

\title{Number Theory Problem: Triangular Numbers and Perfect Squares}
\author{}
\date{}

\begin{document}

\maketitle

\section*{Problem Statement}

A \textbf{triangular number} is a number that can be represented as a triangle with dots. The $n$-th triangular number, denoted $T_n$, is given by the formula:
\[ T_n = \frac{n(n+1)}{2} \]

For example, $T_1 = 1$, $T_2 = 3$, $T_3 = 6$, $T_4 = 10$, etc.

\subsection*{Part (a)}
Show that there exists a positive integer $n$ such that $T_n$ is a perfect square, and find the smallest such $n$.

\subsection*{Part (b)}
For the value of $n$ found in part (a), show that $T_n$ can be expressed as the sum of exactly two triangular numbers.

\subsection*{Part (c)}
Let $k$ be the value of $n$ from part (a). Show that $k$ itself has a special relationship with triangular numbers: specifically, that $k = 2 \cdot T_m$ for some positive integer $m$. Find this value of $m$.

\subsection*{Part (d)}
Using your results from parts (a), (b), and (c), determine the exact numerical value of $T_n$ from part (a).

\vspace{0.5cm}

\textbf{Hint:} You may find it helpful to consider the relationship between triangular numbers and perfect squares, and to look for patterns in the sequence of triangular numbers.

\end{document}
