\documentclass[12pt]{article}
\usepackage{amsmath}
\usepackage{amssymb}
\usepackage{geometry}
\geometry{margin=1in}

\title{Number Theory Problem: Triangular Numbers and Perfect Squares}
\author{}
\date{}

\begin{document}

\maketitle

\section*{Problem Statement}

A \textbf{triangular number} has the form $T_n = \frac{n(n+1)}{2}$.

For example: $T_1 = 1$, $T_2 = 3$, $T_3 = 6$, $T_4 = 10$, $T_5 = 15$, etc.

\textbf{Problem:} What is the smallest number that is both a triangular number and a perfect square?

\vspace{0.5cm}

\textbf{Hint:} You may find it helpful to consider the relationship between triangular numbers and perfect squares, and to look for patterns in the sequence of triangular numbers.

\end{document}
