\documentclass[12pt]{article}
\usepackage{amsmath}
\usepackage{amssymb}
\usepackage{geometry}
\geometry{margin=1in}

\title{Number Theory Problem: Pyramid Numbers and Triangular Numbers}
\author{}
\date{}

\begin{document}

\maketitle

\section*{Problem Statement}

A \textbf{triangular number} has the form $T_n = \frac{n(n+1)}{2}$.

A \textbf{pyramid number} has the form $P_n = T_{T_n}$ (that is, the $T_n$-th triangular number).

For example:
\begin{align}
T_1 &= 1, \quad T_2 = 3, \quad T_3 = 6, \quad T_4 = 10, \ldots \\
P_1 &= T_{T_1} = T_1 = 1 \\
P_2 &= T_{T_2} = T_3 = 6 \\
P_3 &= T_{T_3} = T_6 = 21 \\
P_4 &= T_{T_4} = T_{10} = 55
\end{align}

\textbf{Problem:} What is the smallest number that is both a pyramid number and a triangular number?

\vspace{0.5cm}

\textbf{Hint:} You may find it helpful to consider the relationship between pyramid numbers and perfect squares, and to look for patterns in the sequence of triangular numbers.

\end{document}
