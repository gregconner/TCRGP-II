\documentclass[12pt]{article}
\usepackage{amsmath}
\usepackage{amssymb}
\usepackage{geometry}
\geometry{margin=1in}

\title{Number Theory Problem: Nested Triangular Numbers}
\author{}
\date{}

\begin{document}

\maketitle

\section*{Problem Statement}

A \textbf{triangular number} has the form $T_n = \frac{n(n+1)}{2}$.

For example: $T_1 = 1$, $T_2 = 3$, $T_3 = 6$, $T_4 = 10$, $T_5 = 15$, etc.

Consider the expression $T_{T_n + T_m}$ where $n$ and $m$ are positive integers.

\textbf{Problem:} What is the smallest number that can be expressed as $T_{T_n + T_m}$ for some positive integers $n$ and $m$?

\vspace{0.5cm}

\textbf{Hint:} You may find it helpful to consider small values of $n$ and $m$ first, and to look for patterns in the sequence of triangular numbers.

\end{document}
